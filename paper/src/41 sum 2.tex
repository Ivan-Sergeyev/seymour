\section{The 2-Sum}

The definition on 2-sum is also implemented on the three levels.

% Implementation for Matrices
\begin{leancode}
def matrixSum2 {R : Type*} [Semiring R]
    {Xₗ Yₗ Xᵣ Yᵣ : Type*}
    (Aₗ : Matrix Xₗ Yₗ R) (r : Yₗ → R)
    (Aᵣ : Matrix Xᵣ Yᵣ R) (c : Xᵣ → R) :
    Matrix (Xₗ ⊕ Xᵣ) (Yₗ ⊕ Yᵣ) R :=
  Matrix.fromBlocks
    Aₗ 0 (fun i j => c i * r j) Aᵣ
\end{leancode}
The \texttt{Matrix} level is pretty similar to the one of the 1-sum.
Again, the two given matrices are placed along the main diagonal
of the resulting block matrix.
The resulting two blocks are not both zero, however, as
this time the bottom left matrix contains the
outer product of two given vectors.
The same matrix in a picture:
\[
    \begin{NiceArray}{cccc}[hvlines,right-margin=0.5em,left-margin=0.1em]
        \Block[draw]{2-2}{A_{\ell}} & \phantom{1} & \Block[draw]{2-2}{0} & \phantom{1} \\
        \phantom{1} & \phantom{1} & \phantom{1} & \phantom{1} \\
        \Block[draw]{2-2}{c \otimes r} & \phantom{1} & \Block[draw]{2-2}{A_{r}} & \phantom{1} \\
        \phantom{1} & \phantom{1} & \phantom{1} & \phantom{1} \\
    \end{NiceArray}
\]

% Implementation for Standard Representations
\begin{leancode}
noncomputable def standardReprSum2
    {α : Type*} [DecidableEq α]
    {Sₗ Sᵣ : StandardRepr α Z2} {x y : α}
    (hXX : Sₗ.X ∩ Sᵣ.X = {x})
    (hYY : Sₗ.Y ∩ Sᵣ.Y = {y})
    (hXY : Disjoint Sₗ.X Sᵣ.Y)
    (hYX : Disjoint Sₗ.Y Sᵣ.X) :
    Option (StandardRepr α Z2) :=
  let Aₗ : Matrix (Sₗ.X \ {x}) Sₗ.Y Z2 :=
    Sₗ.B.submatrix Set.diff_subset.elem id
  let Aᵣ : Matrix Sᵣ.X (Sᵣ.Y \ {y}) Z2 :=
    Sᵣ.B.submatrix id Set.diff_subset.elem
  let r : Sₗ.Y.Elem → Z2 := Sₗ.B ⟨x, sorry⟩
  let c : Sᵣ.X.Elem → Z2 := (Sᵣ.B · ⟨y, sorry⟩)
  open scoped Classical in if
    r ≠ 0 ∧ c ≠ 0
  then
    some ⟨
      (Sₗ.X \ {x}) ∪ Sᵣ.X,
      Sₗ.Y ∪ (Sᵣ.Y \ {y}),
      sorry,
      (matrixSum2 Aₗ r Aᵣ c).toMatrixUnionUnion,
      inferInstance,
      inferInstance⟩
  else
    none
\end{leancode}
The \texttt{StandardRepr} level is more complicated.
We first need to slice the last row of the matrix \texttt{Sₗ.B}
and the first column of the matrix \texttt{Sᵣ.B} as the two
separate vectors (\texttt{r} and \texttt{c}), naming the two remaining matrices
\texttt{Aₗ} and \texttt{Aᵣ} respectively.
To identify the special row and the special column
(remember that matrices do not have rows and columns ordered,
as much as we like to draw certain canonical ordering or rows and columns
on paper for helpful visuals), we need to be given
a specific element \texttt{x} in \texttt{Sₗ.X ∩ Sᵣ.X} and
a specific element \texttt{y} in \texttt{Sₗ.Y ∩ Sᵣ.Y} and
promised that there is no other element in any pairwise intersection
among the four indexing sets.
The following picture shows how \texttt{Sₗ.B} and \texttt{Sᵣ.B} are taken apart:
\[
    \texttt{Sₗ.B} = \begin{NiceArray}{cc}[hvlines,right-margin=0.5em,left-margin=0.1em]
        \Block[draw]{2-2}{A_{\ell}} & \phantom{1} \\
        \phantom{1} & \phantom{1} \\
        \Block[draw]{1-2}{r} & \\
    \end{NiceArray}
    , \quad
    \texttt{Sᵣ.B} = \begin{NiceArray}{ccc}[hvlines,right-margin=0.5em,left-margin=0.1em]
        \Block[draw]{2-1}{c} & \Block[draw]{2-2}{A_{r}} & \phantom{1} \\
        & \phantom{1} & \phantom{1} \\
    \end{NiceArray}
\]
Now we know the arguments to be given to the \texttt{Matrix} level.
Again, we convert the output matrix from being indexed by \texttt{Sum} to
being index by set unions, we provide a proof that the resulting
standard representation has row indices and column indices disjoint,
and we check whether the operation is valid\EmDash this time,
the condition is that neither \texttt{r} nor \texttt{c} is a zero vector.

% Implementation for Matroids
\begin{leancode}
def Matroid.IsSum2of {α : Type*} [DecidableEq α]
    (M : Matroid α) (Mₗ Mᵣ : Matroid α) :
    Prop :=
  ∃ S Sₗ Sᵣ : StandardRepr α Z2,
  ∃ x y : α,
  ∃ hXX : Sₗ.X ∩ Sᵣ.X = {x},
  ∃ hYY : Sₗ.Y ∩ Sᵣ.Y = {y},
  ∃ hXY : Disjoint Sₗ.X Sᵣ.Y,
  ∃ hYX : Disjoint Sₗ.Y Sᵣ.X,
  standardReprSum2 hXX hYY hXY hYX = some S
  ∧ S.toMatroid = M
  ∧ Sₗ.toMatroid = Mₗ
  ∧ Sᵣ.toMatroid = Mᵣ
\end{leancode}
This is again a predicate\EmDash when $M$ is a 2-sum of $M_{\ell}$ and $M_{r}$.
