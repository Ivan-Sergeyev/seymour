\section{Theory Underpinning the Formalization}

There are two classical sources presenting the proof of Seymour's decomposition theorem: \cite{Oxley2011} and \cite{Truemper2016}, each with their own advantages and disadvantages.

% Advantages and disadvantages of following Oxley
\citeauthor{Oxley2011} \cite{Oxley2011} develops a general theory of matroids and has a broader focus. It introduces many abstract notions and proves many statements about them, and Seymour's theorem and its dependencies are also stated and proved in terms of these abstract notions alongside many other results. The advantages of following \cite{Oxley2011} would be the higher reusability, generality, and extensibility of the formalization. Indeed, since \cite{Oxley2011} introduces a lot of foundational notions and results, the resulting implementation could serve as the basis for formalization of many other results from classical matroid theory. Moreover, \cite{Oxley2011} is more general than \cite{Truemper2016} in certain aspects, for example, \cite{Truemper2016} defines 1- and 2-sums only for binary matroids, while their definitions in \cite{Oxley2011} do not have this restriction. Finally, it seems that the approach to theory of infinite matroids \cite{Bruhn2013} is more closely aligned with the approach of \cite{Oxley2011} than \cite{Truemper2016}, which might make it easier to generalize formalizations based on the former than the latter to the infinite matroid setting. However, proof formalization following \cite{Oxley2011} would face many challenges. First, the support for matroids in Mathlib at the time we carried out our project was quite limited. Thus a lot of time would be dedicated to developing low-level definitions and results about them, especially in the infinite matroid setting to ensure compatibility with Mathlib. Second, certain intermediate results could prove difficult to formally prove. From our experiments, proving the equivalence of multiple characterizations of regular matroids turned out hard to formalize. Finally, \cite{Oxley2011} leaves many technical steps as exercises for the reader, most crucially leaving out the proof of regularity of 3-sum, and contains many proofs that crucially rely on graph theory, which was not supported in Mathlib. This would make it challenging to convert the proofs to their fully formalized versions.

% Advantages and disadvantages of following Truemper
In contrast, Truemper \cite{Truemper2016} focuses on decomposition and composition of matroids, with Seymour's theorem being one of the most prominent theorems that it builds towards. \citeauthor{Truemper2016} \cite{Truemper2016} more frequently than \citeauthor{Oxley2011} \cite{Oxley2011} utilizes explicit matrix representations in definitions, theorems, and proofs, especially when it comes to 1-, 2-, and 3-sums of regular matroids. Thus, following \cite{Truemper2016} would require implementing fewer intermediate definitions and results to begin working with Seymour's theorem itself. Moreover, Mathlib's support for matrices and linear independence was more extensive than for matroids, so this would allow us to build upon more things that were already available. However, following the approach of \cite{Truemper2016} had several important limitations. As mentioned earlier, it would be less general and potentially less amenable to generalization to the infinite matroid setting than \cite{Oxley2011}. Moreover, faithfully following \cite{Truemper2016} would mean implementing similar definitions and theorems on several levels of abstraction. More specifically, 1-, 2-, and 3-sums would need to be implemented separately for matrices, binary matroids defined by standard representation matrices, and binary matroids in general, and the results about the sums of these objects would need to be proved and propagated accordingly. Last but not least, similar to \cite{Oxley2011}, one would need to fill in the omitted technical details and re-work proofs that could be extremely challenging to formalize directly, especially those involving graph-theoretic arguments.

% Why we chose Truemper over Oxley
Ultimately, we decided to follow the approach of \cite{Truemper2016} over \cite{Oxley2011} for formalizing Seymour's theorem, as it aligned more closely with our goals and values. We aimed to formalize the statement of Seymour's theorem and the proof of the composition direction, so having to implement fewer intermediate definitions and lemmas and being able to use more tools from Mathlib was a big plus. Though we did not mind limiting the generality of our contributions to classical results, our final results go beyond that and hold for matroids of finite rank with potentially infinite ground sets. The completeness of the presentation in \cite{Truemper2016} allowed us to develop a theoretical blueprint, where we fleshed out the technical details, circumvented problematic intermediate results, and streamlined the proofs, especially in the case of 3-sums. %Thus, choosing the approach of \cite{Truemper2016} was the correct decision for our case.

% % omitted: Alternative proof by Geelen and Gerards
% When reviewing the literature on Seymour's theorem, we found that \cite{Geelen2005} presents an alternative approach to a key step in the proof of the decomposition direction. Unfortunately, it heavily relies upon graph theory and planarity, which were not supported in Mathlib at the time of our work, so it would likely be difficult to formalize. Moreover, since we prioritized formalizing the proof of the composition direction of Seymour's theorem, we did not investigate if this alternative approach could make it easier to formally prove the decomposition direction compared to the classical proofs \cite{Oxley2011,Truemper2016}.

% % omitted: Alternative proof by S. R. Kigan (https://arxiv.org/pdf/1403.7757)
% Note: these two papers are tangentially related to our work; they offer alternative proofs for classical Seymour's theorem, we didn't refer to them in our work
